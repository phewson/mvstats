

\section{Preliminaries and exploratory data analysis}

We're going to revisit data we considered in the cluster analysis lab., so you need to load the \texttt{cluster} and the \texttt{flexclust} libraries to make the data available.

We'll start by considering the \texttt{milk} data; you should have done some interesting eda in the cluster lab.   We follow this up by carrying out multidimensional scaling using \texttt{cmdscale} from the \texttt{MASS} library (look at the helpfile):

\begin{verbatim}
library(cluster)
library(flexclust)
library(MASS)
data(milk)
milk.dist <- dist(milk)
milk.pco <- cmdscale(milk.dist)
par(xpd = NA, bty = "n") ## let the labels run past the 
## plotting region, and remove the box
plot(milk.pco, type = "n", main = "PCO representation")
text(milk.pco, row.names(milk.pco), cex = 0.5)
\end{verbatim}

Note that we've produced a blank plot but then drawn text on in the positions of the data with the row names.   Can you make sense of the resulting plot?

If you're very interested, you could even plot the cluster solutions you find when doing cluster analysis against the 2-dimensional representation of the distance matrix:

\begin{verbatim}
milk.hclust <- hclust(milk.dist)
milk.cut <- cutree(milk.hclust, 3)
plot(milk.pco, type = "n", main = "PCO representation")
text(milk.pco, row.names(milk.pco), cex = 0.5, col = milk.cut, pch = milk.cut)
\end{verbatim}


More importantly this week is to consider some diagnostics.


Lets look at the $n-1$ solution, using the delightfully titled \texttt{zapsmall} function to round our eigenvalues to 8 digits (computer arithmetic means we have a lot of negative eigenvalues in the 15th decimal place which we get rid of).   

\begin{verbatim}
milk.pco.24 <- cmdscale(milk.dist, eig = TRUE, k = 24)
evals <- zapsmall(milk.pco.24$eig, digits = 8)
evals
\end{verabtim}

Now, if we want to examine the fit of a $q$ dimensional approximation we use $2 \times n \times \sum_{j=q+1}^{n-1} \lambda_{j}$ (note we are summing all the discarded eigenvalues) to give us: 

\begin{verbatim}
2 * dim(milk)[1] * sum(milk.pco.24$eig[3:24])
\end{verbatim}

You can adjust this last line to see the SS for a 3 dimensional approximation (use \texttt{4:24} in the square brackets) and so on.

And now try to calculate the same thing directly from the projection and from the data.   We already know $\boldsymbol{Delta}$ (although we have to tell \textbf{R} we want a matrix and not a distance matrix.   We can get an estimate of $d$ by taking the distance matrix of the points in the $q$ dimensional approximation that interests us.   Finally, we just want the sum of the differences in the squared distances:

\begin{verbatim}
milk.pco.2 <- cmdscale(milk.dist, eig = TRUE, k = 2)
delta <- as.matrix(milk.dist)
d <- as.matrix(dist(milk.pco.2$points))
sum(as.vector(delta)^2 - as.vector(d)^2)
\end{verbatim}


The other obvious method is just to use the percentage of variance explained / discarded.   Consider dividing the cumulative sum of the eigenvalues by the sum of the eigenvalues:

\begin{verbatim}
cumsum(evals) / sum(evals)
\end{verbatim}


Careful examination should convince you that a 2-dimensional approximation might well be adequate.


\section{Sammon Mapping}

You could try ``Sammon'' mapping on the mvmclass data.   Assuming you've used \texttt{daisy} to generate a distance matrix from the class data (we did this in an earlier lab.)

\begin{verbatim}
mvmclass <- read.csv("class06.csv", row.names = 1)
mvmclass.dist <- daisy(mvmclass)
mvmclass.sammon <- sammon(mvmclass.dist)
\end{verbatim}

If you want to plot the representation, the material you need is under \verb+mvmclass.sammon$points+.

\begin{itemize}
\item How do you determine whether your sammon fit is a good one or not?
\end{itemize}

\section{Further analyses}

In case you wish to follow this from Johnson and Wichern's perspective, two datasets have been placed in the portal are very interesting: \texttt{utilities.csv} and \texttt{USUni.csv}.   The former we saw last time when we were considering clustering.   The latter considers a number of measures on 25 US Universities.   

Another interesting analysis involves the Painters data (see \texttt{?painters} in \texttt{MASS}) - in particular you may like to see whether there any evidence that the first dimension corresponds with a time axis?

\section{Summary}

\fbox{\parbox[c]{0.9\textwidth}{\color{blue}
In some ways, it might have been better for this topic to follow cluster analysis.   It is really a different way of examining the relationships between individuals in a dataset.   By the end of this week, we should be able to:

\begin{itemize}
\item Computer and plot a p.c.o projection of a distance matrix
\item Understand and justify methods for determining whether our low dimension projection is adequate
\item Explain and interpret results, especially if those results are placed relative to a cluster analysis
\item Be able to relate p.c.o. to p.c.a.
\end{itemize}
}}
