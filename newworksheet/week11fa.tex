

\section{Preliminaries and exploratory data analysis}

Consider some data from Manly (you will need to get the file from the portal), OR try this with an example from Johnson and Wichern.   A link to their data is given in the portal.

\begin{Schunk}
\begin{Sinput}
> econ.df <- read.csv("econ.csv", row.names = 1)
> econ <- econ.df[, -10]
> econ.cor <- cor(econ)
> econ.ev <- eigen(econ.cor)
\end{Sinput}
\end{Schunk}

\section{PRINCIPAL COMPONENT EXTRACTION}

Extracts loadings, uniqueness and residuals for 4 factors based on principal component extraction.

\begin{Schunk}
\begin{Sinput}
> loadings4 <- matrix(0, 9, 4)
> for (i in 1:4) {
+     loadings4[, i] <- sqrt(econ.ev$values[i]) * econ.ev$vectors[, 
+         i]
+ }
> LLt <- loadings4 %*% t(loadings4)
> unique <- diag(econ.cor - LLt)
> error <- econ.cor - (LLt + unique)
\end{Sinput}
\end{Schunk}

\begin{itemize}
  \item Now consider how to amend this code to give you a two or three factor solution.
\end{itemize}
    
    
\subsection{RESULTS FROM PRINCIPAL COMPONENT EXTRACTION}

The following labels the loadings for ease of use

\begin{Schunk}
\begin{Sinput}
> row.names(loadings4) <- row.names(econ.cor)
\end{Sinput}
\end{Schunk}
The following gives us the values

\begin{Schunk}
\begin{Sinput}
> loadings4
> unique
> error
\end{Sinput}
\end{Schunk}

The following extracts the eigenvalues, and then calculates the proportion and cumulative proportion of variance explained by each component (for 9 variables)


\begin{Schunk}
\begin{Sinput}
> econ.ev$values
> econ.ev$values/0.09
> cumsum(econ.ev$values/0.09)
\end{Sinput}
\end{Schunk}

The following calculates the communalities

\begin{Schunk}
\begin{Sinput}
> apply(loadings4^2, 1, sum)
\end{Sinput}
\end{Schunk}

The following carries out a rotation (you should also consider promax rotation)

\begin{Schunk}
\begin{Sinput}
> varimax(loadings4)
\end{Sinput}
\end{Schunk}

\section{PRINCIPAL FACTORING WITH ITERATION}

This is rather crude code, but it lets you see what's going on (i.e. try this once and then forget it)

Firstly, an initial estimate of the uniquenesses (can you explain what's going on here?)


\begin{Schunk}
\begin{Sinput}
> r2s <- vector("numeric", 9)
> for (i in 1:9) {
+     y <- econ[, i]
+     x <- econ[, -i]
+     mod <- lm(y ~ as.matrix(x))
+     r2s[i] <- summary(mod)$r.squared
+ }
> unique <- diag(1 - r2s)
> diag(unique)
\end{Sinput}
\end{Schunk}


The following step requires to be repeated until the values of the uniquenesses and the loadings converge (10 - 20 runs will do for now).   Type this code in a text editor and paste it all in as one (when you have it working).


\begin{Schunk}
\begin{Sinput}
> new <- econ.cor - unique
> new.ev <- eigen(new)
> loadings4pf <- matrix(0, 9, 4)
> for (i in 1:4) {
+     loadings4pf[, i] <- sqrt(new.ev$values[i]) * new.ev$vectors[, 
+         i]
+ }
> LLt <- loadings4pf %*% t(loadings4pf)
> unique.f <- econ.cor - LLt
> diag(unique) <- diag(unique.f)
> diag(unique)
> loadings4pf
\end{Sinput}
\end{Schunk}

Repeat that until you get convergence in the loadings estimates.   If
you get convergence in the loadings estimates (one of the pitfalls of
this technique)


\section{MAXIMUM LIKELIHOOD SOLUTION}

For our purposes, we will trust R to do the optimisation.  In other
words, to carry out the analysis we need one line of code!!!!!!!!!!!!)

\begin{Schunk}
\begin{Sinput}
> econ.fact <- factanal(econ, factors = 4, rotation = "none")
> econ.fact
\end{Sinput}
\end{Schunk}

Extract the loadings, calculate the uniquenesses, calculate the residuals

\begin{Schunk}
\begin{Sinput}
> loadml <- loadings(econ.fact)
> loadml
> class(loadml) <- "matrix"
> uniqueml <- econ.fact$uniquenesses
> resid <- econ.cor - (loadml %*% t(loadml) + diag(uniqueml))
> resid
\end{Sinput}
\end{Schunk}

Calculate the communalities

\begin{Schunk}
\begin{Sinput}
> apply(loadml^2, 1, sum)
\end{Sinput}
\end{Schunk}

\section{PLOTTING THE ROTATIONS}

In case it helps, you could try plotting the rotations (the code here
illustrates the pcfa method), with numbers
for the unrotated loadings and letters for the rotated loadings.

\begin{Schunk}
\begin{Sinput}
> plot(loadings4[, c(1:2)], pch = as.character(c(1:9)), xlab = expression(paste(gamma, 
+     "1")), ylab = expression(paste(gamma, "2")), main = "First and second loadings", 
+     xlim = c(-1, 1), ylim = c(-1, 1))
> points(varimax(loadings4)$loadings[, c(1:2)], pch = letters[c(1:9)], 
+     col = "red")
> abline(h = 0)
> abline(v = 0)
\end{Sinput}
\end{Schunk}
\includegraphics{week11fa-plotloadings}


\begin{itemize}
\item Do this for maximum likelihood method
\end{itemize}


\begin{itemize}
  \item How do you interpret the loadings??????
    \end{itemize}




\section{Summary}

\fbox{\parbox[c]{0.9\textwidth}{\color{blue}
We have left factor analysis until last - you should be very clear that despite some superficial similarities it is a very different technique to p.c.a.   By the end of this week we should:

\begin{itemize}
\item Understand the role of factor analysis as a means of explaining the relationships between variables in a given dataset
\item Be able to interpret the results (loadings) of a particular factor analysis, aided or not by rotations
\item To explain the difference between p.c.a. based and maximum likelihood based methods of factor analysis
  \item To interpret the ``signficance'' of solutions of different size, and to justify the use of a particular size solution in all cases
 \end{itemize}



}}




