\chapter{Some R features}

Most of our work in this module will use R software, although we will spend time considering the output of other programs.   R is useful because it lets us work through the various calculations, in addition to having commands for specific analyses.

It is rather useful to save your work somewhere you can find it again later.   Create a subfolder in the u: drive called \texttt{STAT3401} and store all your work for this module in there.   R uses a working directory - it is not always in a sensible place.   You will have to set the working directory once you have opened R each time you run a session.   

\begin{itemize}
\item[] FILE --- CHANGE DIR --[USE DIALOG BOX]
\item[] OR
\item[] \texttt{> setwd("u:/STAT3401")}
\end{itemize}

Make sure you either use the module portal shortcut, or set the working directory \emph{each and every time} you use R.

\emph{Also, you should note that R is case sensitive, therefore R and r are different.}

There is some supplementary R code available in the module portal (mvmmisc.R).   You need to copy this file into your working directory, and load it into R.   It will provide a few additional functions that are not currently available within R.

\begin{verbatim}
> source("mvmmisc.R")
\end{verbatim}


\subsection{Getting help}

R has an extensive help system.   If you know what you're looking for.   The following commands give you the help page for a particular function:

\begin{verbatim}
> help(seq)
> ?seq
\end{verbatim}



If you've a vague idea, \texttt{apropos} will prompt you for other terms that might be relevant, have a look at \texttt{apropos(seq)}.


\texttt{help.search()} (note the quotation marks) will do a search for any help files containing the word you are looking for, have a look at \texttt{help.search(``seq'')}


And finally, typing \texttt{help.start()} will open up a web-browser interface to the R help system.


\subsection{Tidying objects}

In a while, we'll look at creating objects within R.   All these objects get stored in your workspace in one file called \texttt{.Rdata}.   If you don't take care, this will soon use up all your available file space.   R will only ask you if you want to save your workspace at the end of any session.   If you want to save any work in the meantime, type \texttt{save.image()} at intervals.

Type \texttt{objects()} to get a list of any objects that are stored in your workspace.   

If you wish to remove an objects, called spot, type \texttt{rm(spot)}.

You can quit R by typing \texttt{q()}, and you will be prompted as to whether you wish to save your workspace.   If you say yes, all your work that session will be saved.   If you say no, R will save the workspace you started the session with.

\subsection{R as an interactive calculator}


At it's simplest, R just works in interactive mode - you type in a command, it gives you an answer:

\begin{verbatim}
> 2 + 3
> 2 - 3
> 2 * 3
> 2/3
> 2^3
\end{verbatim}

You might like to check that the operator precedence works correctly.

\begin{verbatim}
> 4^2 - 3 * 2
> (4^2) - (3 * 2)
> -2 - -3
> -2 - -3
> 1 - 6 + 4
> 2^-3
\end{verbatim}



R has a fairly full range of mathematical functions, e.g.

\begin{verbatim}
> log(100)
> log(100, base = 10)
> log(100, b = 10)
\end{verbatim}

If you're not sure what's going on here, type \texttt{?log} to check.

\subsection{Data objects}

R stores data in a range of objects.   For example:

\begin{verbatim}
> x <- 1
\end{verbatim}

The less than, and the minus sign used together create an ``assignment operator'', we have assigned the value of 1 to x.   If we store our workspace, R will remember that x has a value of 1.   This isn't terribly interesting, in practice we need to store rather more useful information.   If we enter:

\begin{verbatim}
> x <- c(5.4, 5.7, 3.8, 4.1, 4.4, 6.1, 9.8)
> names <- c("Suttons", "Bass", "Boddingtons", "Courage", 
      "Stella", "Fosters", "Teignworthy")
\end{verbatim}

we have a range of values for x and a set of names.   We can store these data together in a data frame, called spot:

\begin{verbatim}
> spot <- data.frame(x, names)
> spot
> spot$x
> spot$names
> barplot(spot$x, names.arg = as.character(spot$names))
\end{verbatim}

Having created the data frame, we can extract individual columns using the dollar sign between the name of the data frame and the name of the column.

In practice, we will either use data which is already available within R as a data frame, or we will load data from Excel type csv files (comma separated value) - where it will be imported as a data frame.   The rather useful thing about R is that we can have as many of these data frames in our workspace as we want - but watch your housekeeping!

Data can also be stored in matrices (arrays) as well as lists in R.   We will look much harder at matrices next week!


%%% Local Variables: ***
%%% mode:latex ***
%%% TeX-master: "book.tex"  ***
%%% End: ***