\chapter{Multivariate normality}
\label{mvn}

The multivariate normal distribution remains central to most work concerning multivariate continuous data.   Experience has suggested that it is usually at least an acceptable approximation, and of course one usually has recourse to the central limit theorem.   Although we will examine a few common alternatives, it is perhaps in the field of robust statistics where it's use has been modified most.

\section{Expectations and moments of continuous random functions}
\label{meancovar}

The mean and covariance can be defined in a similar way to to the univariate context

\begin{definition}
\label{def:mvexpectation}

Given a multivariate distribution function for $p$ random variables $\boldsymbol{x} = (x_{1}, x_{2}, \ldots, x_{p})$ taking the form $P(\boldsymbol{x}\in A) = \int_{A} f(\boldsymbol{x})d\boldsymbol{x}$, expectation can be defined as:
\begin{equation}
\label{eq:mvexp}
E(g(\boldsymbol{x})) = \int_{\mathscr{R^{p}}} g(\boldsymbol{x})f(\boldsymbol{x})d\boldsymbol{x}
\end{equation}

which gives rise to moments
\begin{equation}
E(\boldsymbol{x}) = \boldsymbol{\mu}
\end{equation}

as well as moment generating functions:

\begin{equation}
M_{x}\boldsymbol{t} = E e^{\boldsymbol{x}^{T}\boldsymbol{t}},
\end{equation}
cumulants:
\begin{equation}
K_{x}\boldsymbol{t} = \log( M_{x}\boldsymbol{t}), 
\end{equation}
and the characteristic function:
\begin{equation}
\phi_{x}\boldsymbol{t} = E( E e^{i\boldsymbol{x}^{T}\boldsymbol{t}})
\end{equation}

These generating functions have analogous properties to their univariate counterparts.

\end{definition}


\section{Multivariate normality}
\label{mvndetail}

\begin{definition}
\label{def:mvnpdf}

If $\boldsymbol{x} = (x_{1}, x_{2}, \ldots, x_{p})$ is a $p$ dimensional vector of random variables, then $\boldsymbol{y}$ has a multivariate normal distribution if its density function is:
\begin{displaymath}
\label{eq:mvnpdf}
f(\boldsymbol{x}) = \frac{1}{(2 \pi)^{p/2} |\Sigma|^{1/2}}
e^{-(\boldsymbol{x - \mu})^{T}\boldsymbol{\Sigma}^{-1} (\boldsymbol{x
    - \mu})};
\end{displaymath}
for $ -\infty < y_{ij} < \infty, j = 1, 2, \ldots, p$


\end{definition}


And it can be shown that $E(\boldsymbol{x}) = \boldsymbol{\mu}$, and that $Var(\boldsymbol{x}) = \boldsymbol{\Sigma}$, hence we can use the notation:

\begin{displaymath}
\boldsymbol{y} \sim MVN_{p}(\boldsymbol{\mu}, \boldsymbol{\Sigma})
\end{displaymath}

Finding the maximum likelihood estimators for $\boldsymbol{\mu}$ and $\boldsymbol{\Sigma}$ is not trivial, there are perhaps at least three derivations.   We briefly recap results from one of the more popular derivations here.

\begin{theorem}
\label{th:mvnlike}

If $\boldsymbol{x} = (x_{1}, x_{2}, \ldots, x_{p})$ is a $p$ dimensional vector of random variables representing a sample from $MVN_{p}(\boldsymbol{\mu}, \boldsymbol{\Sigma})$, then the log-likelihood function can be given by $\mathscr{l}$ as follows:
\begin{equation}
\label{eq:mvnlike}
\mathscr{l}(\boldsymbol{\mu}, \boldsymbol{\Sigma}| \boldsymbol{x}) = 
- \frac{np}{2} \log(2 \pi) - \frac{n}{2} \log \lvert \boldsymbol{\Sigma} \rvert -
\frac{1}{2} \sum_{i=1}^{n} \left( (\boldsymbol{x}_{i} - \boldsymbol{\mu})^{T} \boldsymbol{\Sigma}^{-1}(\boldsymbol{x}_{i} - \boldsymbol{\mu}) \right)
\end{equation}

which can be rewritten as:

\begin{displaymath}
\label{eq:mvnlike}
\mathscr{l}(\boldsymbol{\mu}, \boldsymbol{\Sigma}| \boldsymbol{x}) = 
- \frac{np}{2} \log(2 \pi) - \frac{n}{2} \log \lvert \boldsymbol{\Sigma} \rvert -
\frac{1}{2} trace \left( \boldsymbol{\Sigma}^{-1} \sum_{i=1}^{n} (\boldsymbol{x}_{i} - \boldsymbol{\mu})^{T}(\boldsymbol{x}_{i} - \boldsymbol{\mu}) \right)
\end{displaymath}

adding and subtracting $\bar{\boldsymbol{x}}$ from each of the two brackets on the right, and setting $\boldsymbol{A} = \sum_{i=1}^{n}(\boldsymbol{x}_{i} - \bar{\boldsymbol{x}})(\boldsymbol{x}_{i} - \bar{\boldsymbol{x}})^{T}$ allows this to be rewritten as: 

\begin{displaymath}
\label{eq:mvnlike}
%\mathscr{l}(\boldsymbol{\mu}, \boldsymbol{\Sigma}| \boldsymbol{x}) = 
- \frac{np}{2} \log(2 \pi) - \frac{n}{2} \log \lvert \boldsymbol{\Sigma} \rvert -
\frac{1}{2} trace \left( \boldsymbol{\Sigma}^{-1} \boldsymbol{A} + n \boldsymbol{\Sigma}^{-1} (\bar{\boldsymbol{x}} - \boldsymbol{\mu})(\bar{\boldsymbol{x}} - \boldsymbol{\mu})^{T} \right)
\end{displaymath}

The way to obtaining the maximum likelihood estimators for $\boldsymbol{\mu}$ is now fairly clear.   As $\boldsymbol{\Sigma}$ is positive definite, we require $(\bar{\boldsymbol{x}} - \boldsymbol{\mu})(\bar{\boldsymbol{x}} - \boldsymbol{\mu})^{T}$ to be greater than or equal to zero, hence $ \hat{\boldsymbol{\mu}} = \bar{\boldsymbol{x}}$ maximises the likelihood for all positive definite $\boldsymbol{\Sigma}$

A number of derivations for Sigma are possible, essentially we need to minimise:
\begin{equation}
\mathscr{l}(\bar{\boldsymbol{x}}, \boldsymbol{\Sigma}) = \log \lvert \boldsymbol{\Sigma} \rvert + trace( \boldsymbol{\Sigma}^{-1} \boldsymbol{S}).
\end{equation}
with respect to $\boldsymbol{\Sigma}$.   This can be derived as a minimisation of $\mathscr{l}(\bar{\boldsymbol{x}},\boldsymbol{\Sigma})  - \log( \lvert \boldsymbol{S} \rvert) = trace( \boldsymbol{\Sigma}^{-1} \boldsymbol{S}) - \log \lvert \boldsymbol{\Sigma}^{-1} \boldsymbol{S} \rvert$.   If $\boldsymbol{S}^{1/2}$ is the positive definite symmetric square root of $\boldsymbol{S}$,  $trace( \boldsymbol{\Sigma}^{-1} \boldsymbol{S}) =  trace( \boldsymbol{S}^{1/2} \boldsymbol{\Sigma}^{-1} \boldsymbol{S}^{1/2})$, but given that $\boldsymbol{A} =  \boldsymbol{S}^{1/2} \boldsymbol{\Sigma}^{-1} \boldsymbol{S}^{1/2}$ is a symmetric matrix we know that $trace(\boldsymbol{A}) = \sum_{j=1}^{p} \lambda_{i}$ and $\lvert \boldsymbol{A} \rvert = \prod_{j=1}^{p} \lambda_{i}$ where $\lambda_{1}, \ldots, \lambda_{p}$ are the eigenvalues of $\boldsymbol{A}$, which must all be positive (because $\boldsymbol{A}$ is positive definite).   Hence we wish to minimise:
\begin{equation}
\mathscr{l}(\bar{\boldsymbol{x}},\boldsymbol{\Sigma})  - \log( \lvert \boldsymbol{S} \rvert) =  \sum_{j=1}^{p} \lambda_{i} - \log  \prod_{j=1}^{p} \lambda_{i}
\end{equation}
and given that $f(z) = z - \log(z)$ takes a unique mininum at $z=1$ we wish to find a matrix where all the eigenvalues equal 1, i.e. the identity matrix.   Consequently we wish to find:
\begin{equation}
 \boldsymbol{S}^{1/2} \boldsymbol{\Sigma}^{-1} \boldsymbol{S}^{1/2} = \boldsymbol{I}
\end{equation}
hence $\boldsymbol{S}$ is the maximum likelihood estimator of $\boldsymbol{\Sigma}$.   Do note that this requires $n>p$.


\end{theorem}

\subsection{\textbf{R} estimation}

\verb+cov()+ and \verb+var()+ (equivalent calls) both give the unbiased estimate for the variance-covariance matrix, i.e. $\frac{1}{n-1} \sum_{i=1}^{n}(\boldsymbol{x}_{i} - \bar{\boldsymbol{x}})^{T}(\boldsymbol{x}_{i} - \bar{\boldsymbol{x}})$.   It is worth nothing that by default, \verb+cov.wt()+ (which will do a few other things as well) uses the divisor $\frac{1}{n}$


%\section{Skewness}
%\label{skewness}

%\section{Outliers}
%\label{outliers}

%\section{Missing Data}
%\label{missing}

\section{Transformations}
\label{transform}

\cite{Box+Cox:1964} modified earlier proposals by \cite{Tukey:1957} to yield the following transformation:

\begin{displaymath}
x^{(\lambda)} = \left\{ \begin{array}{lll} 
  \frac{x^{\lambda}-1}{\lambda}, && \lambda \neq 0,\\
  \log x, && \lambda = 0\ \mbox{and} x > 0.
\end{array} \right. 
\end{displaymath}


\begin{equation}
\mathscr{L}(\mu, \sigma^{2}, \lambda | x^{(\lambda)}) \propto (2 \pi \sigma^{2})^{-n/2} 
\exp(-\sum_{i=1}^{n} \frac{(x_{i}^{(\lambda)} - \mu)^{2}}{2 \sigma^{2}}
\prod_{i=1}^{n}x_{i}^{\lambda-1}
\end{equation}

%the last term being the Jacobian

If $\lambda$ is fixed, this likelihood is maximised at:

\begin{equation}
\bar{x}^{(\lambda)} = \frac{1}{n} \sum_{i=1}{n} x_{i}^{(\lambda)}
\end{equation}
and 
\begin{equation}
s^{2}(\lambda) = \frac{1}{n} \sum_{i=1}^{n}(x_{i}^{(\lambda)} - \bar{x}^{(\lambda)})^{2}
\end{equation}

which means that the value maximising the log-likelihood is proportional to:

\begin{equation}
\mathscr{L_{max}}(\lambda) = -\frac{n}{2} \log s^{2}(\lambda) + (\lambda - 1) \sum_{i=1}^{n} \log x_{i}
\end{equation}

It is possible to apply this transformation to each variable in turn to obtain marginal normality
\cite{Gnanadesikan:1977} argues that this can be used satisfactorily in many cases.

However, it may be preferable to carry out a multivariate optimisation of the transformation parameters.


A range of tranformations have been considered for multivariate data, mainly of the Box-Cox type.   If the variables $\boldsymbol{y} = (y_{1}, y_{2}, \ldots, y_{p}$ are are smooth transformation of $\boldsymbol{x}$, the frequency function for $\boldsymbol{y}$ can be given by:
\begin{displaymath}
g(\boldsymbol{y}) = f(\boldsymbol{x}(\boldsymbol{y}))\lvert \frac{\partial \boldsymbol{x}}{\partial \boldsymbol{y}} \rvert
\end{displaymath}
where $\boldsymbol{x}(\boldsymbol{y})$ is $\boldsymbol{x}$ expressed in terms of the elements of $\boldsymbol{y}$, and $J = \lvert \frac{\partial \boldsymbol{x}}{\partial \boldsymbol{y}} \rvert$ is the Jacobian % determinant of partial derivatives
which ensures the density is mapped correctly. 

$\prod_{j=1}^{p} \prod_{i=1}^{n} x_{ij}^{\lambda_{j}-1}$


\begin{equation}
\mathscr{L}(\boldsymbol{\mu, \Sigma, \lambda} | \boldsymbol{X^{(\lambda)}}) \propto -\frac{n}{2} \log \lvert \boldsymbol{\Sigma} \rvert - 
\frac{1}{2} tr (\boldsymbol{\Sigma}^{-1} 
(\boldsymbol{X}^{(\boldsymbol{\lambda})} - \boldsymbol{1}\boldsymbol{\mu}^{T})^{T}
(\boldsymbol{X}^{(\boldsymbol{\lambda})} - \boldsymbol{1}\boldsymbol{\mu}^{T})
+ sum_{j=1}^{p} \left( (\lambda_{j} - 1) \sum_{i=1}^{n} \log x_{ij} \right)
\end{equation}

%the last term being the Jacobian

If $\lambda$ is fixed, this likelihood is maximised at:

\begin{equation}
\bar{x}^{(\lambda)} = \frac{1}{n}  \boldsymbol{1}^{T} \boldsymbol{X}
\end{equation}
and 
\begin{equation}
s^{2}(\lambda) = (\boldsymbol{X}^{(\boldsymbol{\lambda})} - \boldsymbol{1}\boldsymbol{\mu}^{T})^{T}
(\boldsymbol{X}^{(\boldsymbol{\lambda})} - \boldsymbol{1}\boldsymbol{\mu}^{T})
\end{equation}

which means that the value maximising the log-likelihood is proportional to:

\begin{equation}
\mathscr{L_{max}}(\lambda) = -\frac{n}{2} \log \lvert \hat{\boldsymbol{\Sigma}} \rvert + \sum_{j=1}^{p} \left( (\lambda_{j} - 1) \sum_{i=1}^{n} \log x_{ij} \right)
\end{equation}






%%% Local Variables: ***
%%% mode:latex ***
%%% TeX-master: "../book.tex"  ***
%%% End: ***