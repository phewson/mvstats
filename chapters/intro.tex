\chapter*{Books}

Many of the statistical analyses encountered to date consist of a \emph{single response variable} and \emph{one or more explanatory variables}.  In this latter case, \emph{multiple regression}, we regressed a single response (dependent) variable on a number of explanatory (independent) variables.   This is occasionally referred to as ``multivariate regression'' which is all rather unfortunate.   There isn't an entirely clear ``canon'' of what is a multivariate technique and what isn't (one could argue that discriminant analysis involves a single dependent variable).   However, we are going to consider the simultaneous analysis of a number of related variables.   We may approach this in one of two ways.   The first group of problems relates to classification, where attention is focussed on individuals who are more alike.   In unsupervised classification (cluster analysis) we are concerned with a range of algorithms that at least try to identify individuals who are more alike if not to distinguish clear groups of individuals.   There are also a wide range of scaling techniques which help us visualise these differences in lower dimensionality.   In supervised classification (discriminant analysis) we already have information on group membership, and wish to develop rules from the data to classify future observations.  

The other group of problems concerns inter-relationships between variables.   Again, we may be interested in lower dimension that help us visualise a given dataset.   Alternatively, we may be interested to see how one group of variables is correlated with another group of variables.   Finally, we may be interested in models for the interrelationships between variables.


%Here are a few examples of multivariate analyses.   

%(a) Skulls have been recovered from an ancient burial ground.   These may come from one group, or a mixture of ethnic groups slaughtered in battle.   If we make a number of measurements on the skulls can we estimate the number of groups inovled.

%(b) Data are identified on business activity, such as national income, rate of interest, unemployment.   Can we generate a single index of ``business activity''.

%(c) A social scientist wishes to confirm the concept of ``self esteem'', based on student survey with 10 questions on attitudes.   Are the answers consistent with the existence of such a non-measurable concept.

%\subsection{Objectives}

%\begin{itemize}
%\item Grouping or sorting - objects, individuals or variables are arranged into similar clusters or groups
%\item Prediction or decision making - number of groups is known but rule(s) are required to classify individuals or objects into such groups
%\item Data reduction or simplification - relatively few indices are constructed from many measured variables
%\item Hypotheses are postulated, tested, investigated - either by analogy to univariate hypotheses or an attempt to reveal underlying influences or latent structure
%\end{itemize}

There is a reasonably established range of multivariate techniques which we will cover here.   Introductory level books include  \cite{Johnson+Wichern:2002}
%\cite{Afifi+Clark:1990} \cite{Chatfield+Collins:1980} \cite{Dillon+Goldstein:1984} \cite{Everitt+Dunn:1991} \cite{Flury+Riedwyl:1988} \cite{Johnson:1998}
%\cite{Kendall:1975} \cite{Hair+etal:1995} \cite{Hair+etal:1998} \cite{Manly:1994}

Intermediate level books include:
\cite{Flury:1997} (My personal favourite)
%\cite{Gnanadesikan:1997}
%\cite{Harris:1985} \cite{Krzanowski:2000} \cite{Krzanowski+Marriott:1994,Krzanowski+Marriott:1994II} \cite{Rencher:2002} \cite{Morrison:2005}
%\cite{Seber:1984} \cite{Timm:1975}


More advanced books:
%\cite{Anderson:1984} (I think there's a 2003 edition now) \cite{Bilodeau+Brenner:1999} \cite{Giri:2003} \cite{Mardia+etal:1979} \cite{Muirhead:1982}
\cite{Press:1982} \cite{Srivastava+Carter:1983}.


If you hunt around you will notice that the ``canon'' of multiviate methods does vary.   Some people include contingency tables and log-linear modelling, others exclude Cluster analysis.   Given that multivariate methods are particularly common in applied areas such Ecology and Psychology, it is worth a browse around library sections containing books aimed at these subjects.   It is quite possible that they will have very readable descriptions of particular techniques.   In particular, there seems to be a copious collection of books aimed at Psychologists on Factor Analysis - but there are a few caveats in the way they approach this technique and these books need to be read in the light of cautions given during lectures.



%%% Local Variables: ***
%%% mode:latex ***
%%% TeX-master: "../book.tex"  ***
%%% End: ***
