\begin{theorem}
\label{th:propexpl}

We denote our estimate of the proportion of variation explained by $\pi$:
\begin{displaymath}
\pi = f(\lambda) = \frac{ \sum_{i=1}^{q} \lambda_{i} }{ \sum_{i=1}^{p} \lambda_{i} }.
\end{displaymath}

If we also consider the corresponding sum of squares:
\begin{displaymath}
\zeta = \frac{\sum_{i=1}^{q} \lambda_{i}^{2}}{ \sum_{i=1}^{p} \lambda_{i}^{2}}
\end{displaymath}

Under conditions of multivariate normality, we can obtain an estimate of the variance associated $\pi$, the proportion of variance explained as follows:
\begin{equation}
\eta^{2} = \frac{2 trace(\Sigma)}{(n-1)  (trace(\Sigma))^{2}} = \pi^{2} - 2 \zeta \pi + \zeta^{2}
\end{equation}
\end{theorem}
\textbf{Proof}: \cite{Sugiyama+Tong:1976} and [page 454] \cite{Kshinaragar:1972}
