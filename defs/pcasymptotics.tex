\begin{theorem}
\label{th:pcasymptotics}
The asymptotic distribution of eigenvalues and eigenvectors can be expressed as follows:
\begin{equation}
\label{pcasymptotics}
\sqrt{n}(\boldsymbol{\ell} - \boldsymbol{\lambda}) \sim MVN(\boldsymbol{0}, 2\boldsymbol{\Lambda}^{2})
\end{equation}
%\begin{equation}
%\sqrt{n} \frac{\mathscr{l}_{i} - \lambda_{i}}{\sqrt{2}\mathscr{l}_{i}} \to^\infty N(0,1)
%\end{equation}
where $\Lambda$ is the diagonal matrix of eigenvalues.   This can be expressed equivalently as:
\begin{equation}
\label{pcaasymptoticslog}
\sqrt{n}\left(\log(\boldsymbol{\ell}) - \log(\boldsymbol{\lambda})\right) \sim MVN(\boldsymbol{0}, 2)
\end{equation}

 and 
\begin{equation}
\sqrt{n}(\hat{\boldsymbol{e}}_{i} - \boldsymbol{e}_{i}) \sim MVN(\boldsymbol{0}, \boldsymbol{E}_{i})
\end{equation}
where $\boldsymbol{E}_{i} = \lambda_{i} \sum_{k = 1, k \neq i}^{p} \frac{\lambda_{k}}{(\lambda_{k} - \lambda_{i})^{2}} \boldsymbol{e}_{k}\boldsymbol{e}^{T}$
\end{theorem}
Proof: The proof for this has been restated in \cite{Flury:1988}.   These properties were established following work by \cite{Anderson:1963} and \cite{Girshink:1939}.
